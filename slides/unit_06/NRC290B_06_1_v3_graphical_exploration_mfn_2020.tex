\documentclass[ignorenonframetext,t]{beamer}
\setbeamertemplate{caption}[numbered]
\setbeamertemplate{caption label separator}{: }
\setbeamercolor{caption name}{fg=normal text.fg}
\beamertemplatenavigationsymbolsempty
\usepackage{lmodern}
\usepackage{amssymb,amsmath}
\usepackage{ifxetex,ifluatex}
\usepackage{fixltx2e} % provides \textsubscript
\ifnum 0\ifxetex 1\fi\ifluatex 1\fi=0 % if pdftex
  \usepackage[T1]{fontenc}
  \usepackage[utf8]{inputenc}
\else % if luatex or xelatex
  \ifxetex
    \usepackage{mathspec}
  \else
    \usepackage{fontspec}
  \fi
  \defaultfontfeatures{Ligatures=TeX,Scale=MatchLowercase}
\fi
\usecolortheme{spruce}
\usefonttheme{serif}
% use upquote if available, for straight quotes in verbatim environments
\IfFileExists{upquote.sty}{\usepackage{upquote}}{}
% use microtype if available
\IfFileExists{microtype.sty}{%
\usepackage{microtype}
\UseMicrotypeSet[protrusion]{basicmath} % disable protrusion for tt fonts
}{}
\newif\ifbibliography
\hypersetup{
            pdftitle={Week 5: Exploring Data},
            colorlinks=true,
            linkcolor=Maroon,
            citecolor=Blue,
            urlcolor=blue,
            breaklinks=true}
\urlstyle{same}  % don't use monospace font for urls

% Prevent slide breaks in the middle of a paragraph:
\widowpenalties 1 10000
\raggedbottom

\AtBeginPart{
  \let\insertpartnumber\relax
  \let\partname\relax
  \frame{\partpage}
}
\AtBeginSection{
  \ifbibliography
  \else
    \let\insertsectionnumber\relax
    \let\sectionname\relax
    \frame{\sectionpage}
  \fi
}
\AtBeginSubsection{
  \let\insertsubsectionnumber\relax
  \let\subsectionname\relax
  \frame{\subsectionpage}
}

\setlength{\parindent}{0pt}
\setlength{\parskip}{6pt plus 2pt minus 1pt}
\setlength{\emergencystretch}{3em}  % prevent overfull lines
\providecommand{\tightlist}{%
  \setlength{\itemsep}{0pt}\setlength{\parskip}{0pt}}
\setcounter{secnumdepth}{0}
\usepackage{multicol}
\usepackage{tikz}
\usepackage{tikzpagenodes}
\definecolor{grn}{rgb}{0.0, 0.2, 0.0}
\usepackage{tabto}
\usepackage{verbatim}
\usepackage{amsmath}
\usepackage{mathtools}
\usepackage{graphicx}
\definecolor{OG}{RGB}{0,64,8}
\definecolor{LG}{RGB}{0,102,51}
\definecolor{myRed}{RGB}{228,26,28}
\definecolor{myBlue}{RGB}{55,126,184}
\definecolor{myGreen}{RGB}{77,175,74}
\definecolor{myPurple}{RGB}{152,78,163}
\setbeamercolor{itemize item}{fg=white!0!LG}
\setbeamercolor{enumerate item}{fg=white!0!LG}
\setbeamercolor{enumerate subitem}{fg=white!70!LG}
\setbeamercolor{itemize subitem}{fg=white!70!LG}
\setbeamercolor{itemize subsubitem}{fg=white!70!LG}
\setbeamercolor{navigation symbols}{fg=white!70!LG, bg=white!70!LG}
\usepackage{inputenc}
\usepackage{booktabs}
\usepackage{caption}
\usetikzlibrary{patterns,arrows,decorations.pathreplacing}

\title{Week 5: Exploring Data}
\subtitle{Session 1}
\date{Spring 2020}

\begin{document}
\frame{\titlepage}

\begin{frame}

\newcommand\nnn{\includegraphics[height=0.25in]{iClicker_logo.png}}


\end{frame}

\begin{frame}[fragile]{iClicker Question \textcolor{blue}{1}}

Which of the following should I use to read the file \texttt{mydata.csv}
into a data frame called \texttt{dat} in R?

\begin{enumerate}[A]
\item \texttt{dat = read.csv(mydata.csv)}
\item \texttt{read.csv(mydata.csv)}
\item \texttt{dat = read.csv("mydata.csv")}
\item \texttt{read.csv(mydata.csv, row.names = 1)}
\item \texttt{read.csv("mydata.csv")}
\end{enumerate}

\nnn

\end{frame}

\begin{frame}{iClicker Question \textcolor{blue}{2}}

Which symbol do we use to represent the \textbf{sample mean}?

\begin{enumerate}[A]
\item $\sigma$
\item $\bar s$
\item $\bar{x}$
\item $\mu$
\item $\bar{m}$
\end{enumerate}

\vfill

\begin{tikzpicture}[remember picture,overlay]
\node[xshift=-2cm,yshift=1cm] at (current page.south east)
{\includegraphics[height=0.25in]{iClicker_logo.png}};
\end{tikzpicture}

\end{frame}

\begin{frame}{iClicker Question \textcolor{blue}{3}}

Which symbol do we use to represent the \textbf{population mean}?

\begin{enumerate}[A]
\item $\sigma$
\item $\bar s$
\item $\bar{x}$
\item $\mu$
\item $\bar{m}$
\end{enumerate}

\vfill

\begin{tikzpicture}[remember picture,overlay]
\node[xshift=-2cm,yshift=1cm] at (current page.south east)
{\includegraphics[height=0.25in]{../slide_images/iClicker_logo.png}};
\end{tikzpicture}

\end{frame}

\begin{frame}{iClicker Question \textcolor{blue}{4}}

Which plot type is most appropriate to show the \textbf{distribution} of
a set of measurements?

\begin{enumerate}[A]
\item scatterplot
\item boxplot
\item barchart
\item histogram
\item pie chart
\end{enumerate}

\vfill

\begin{tikzpicture}[remember picture,overlay]
\node[xshift=-2cm,yshift=1cm] at (current page.south east)
{\includegraphics[height=0.25in]{../slide_images/iClicker_logo.png}};
\end{tikzpicture}

\end{frame}

\begin{frame}[fragile]{iClicker Question \textcolor{blue}{5}}

Which of the following lines of code will make a scatterplot of the
dataframe with length on the x-axis and mass on the y-axis?

\begin{verbatim}
##      length     width mass
## 1 0.7209039 2.8777226   37
## 2 0.8757732 1.6052823   29
## 3 0.7609823 0.4713699   19
\end{verbatim}

\begin{enumerate}[A]
\item \texttt{plot(dat\$mass, dat\$length)}
\item \texttt{scatter(dat\$length, dat\$mass)}
\item \texttt{boxplot(dat\$length, dat\$mass, type = "p")}
\item \texttt{dotplot(dat\$length, dat\$mass)}
\item \texttt{plot(dat\$length, dat\$mass)}
\end{enumerate}

\vfill

\begin{tikzpicture}[remember picture,overlay]
\node[xshift=-2cm,yshift=1cm] at (current page.south east)
{\includegraphics[height=0.25in]{../slide_images/iClicker_logo.png}};
\end{tikzpicture}

\end{frame}

\end{document}
