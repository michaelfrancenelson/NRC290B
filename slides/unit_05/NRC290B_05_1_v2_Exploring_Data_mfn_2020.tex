\documentclass[ignorenonframetext,]{beamer}
\setbeamertemplate{caption}[numbered]
\setbeamertemplate{caption label separator}{: }
\setbeamercolor{caption name}{fg=normal text.fg}
\beamertemplatenavigationsymbolsempty
\usepackage{lmodern}
\usepackage{amssymb,amsmath}
\usepackage{ifxetex,ifluatex}
\usepackage{fixltx2e} % provides \textsubscript
\ifnum 0\ifxetex 1\fi\ifluatex 1\fi=0 % if pdftex
  \usepackage[T1]{fontenc}
  \usepackage[utf8]{inputenc}
\else % if luatex or xelatex
  \ifxetex
    \usepackage{mathspec}
  \else
    \usepackage{fontspec}
  \fi
  \defaultfontfeatures{Ligatures=TeX,Scale=MatchLowercase}
\fi
\usetheme[]{default}
\usecolortheme{spruce}
\usefonttheme{serif}
% use upquote if available, for straight quotes in verbatim environments
\IfFileExists{upquote.sty}{\usepackage{upquote}}{}
% use microtype if available
\IfFileExists{microtype.sty}{%
\usepackage{microtype}
\UseMicrotypeSet[protrusion]{basicmath} % disable protrusion for tt fonts
}{}
\newif\ifbibliography
\hypersetup{
            pdftitle={Week 5: Exploring Data},
            pdfborder={0 0 0},
            breaklinks=true}
\urlstyle{same}  % don't use monospace font for urls
\usepackage{graphicx,grffile}
\makeatletter
\def\maxwidth{\ifdim\Gin@nat@width>\linewidth\linewidth\else\Gin@nat@width\fi}
\def\maxheight{\ifdim\Gin@nat@height>\textheight0.8\textheight\else\Gin@nat@height\fi}
\makeatother
% Scale images if necessary, so that they will not overflow the page
% margins by default, and it is still possible to overwrite the defaults
% using explicit options in \includegraphics[width, height, ...]{}
\setkeys{Gin}{width=\maxwidth,height=\maxheight,keepaspectratio}

% Prevent slide breaks in the middle of a paragraph:
\widowpenalties 1 10000
\raggedbottom

\AtBeginPart{
  \let\insertpartnumber\relax
  \let\partname\relax
  \frame{\partpage}
}
\AtBeginSection{
  \ifbibliography
  \else
    \let\insertsectionnumber\relax
    \let\sectionname\relax
    \frame{\sectionpage}
  \fi
}
\AtBeginSubsection{
  \let\insertsubsectionnumber\relax
  \let\subsectionname\relax
  \frame{\subsectionpage}
}

\setlength{\parindent}{0pt}
\setlength{\parskip}{6pt plus 2pt minus 1pt}
\setlength{\emergencystretch}{3em}  % prevent overfull lines
\providecommand{\tightlist}{%
  \setlength{\itemsep}{0pt}\setlength{\parskip}{0pt}}
\setcounter{secnumdepth}{0}
\input{C:/Users/michaelnelso/git/NRC290B/css/headers_tikz.yaml}

\title{Week 5: Exploring Data}
\subtitle{Session 1}
\date{Spring 2020}

\begin{document}
\frame{\titlepage}

\begin{frame}{iClicker quiz: Question \textcolor{blue}{1}}

Which of the following appears Normally-distributed?

\includegraphics{NRC290B_05_1_v2_Exploring_Data_mfn_2020_files/figure-beamer/unnamed-chunk-1-1.pdf}

\begin{multicols}{2}
\null \vfill
\vfill \null
\columnbreak
\includegraphics[width = 0.35\textwidth]{../slide_images/iClicker_logo.png}

\end{multicols}

\end{frame}

\begin{frame}{iClicker quiz: Question \textcolor{blue}{2}}

What is the \textbf{median} value of the following sequence of numbers?

\begin{center}
\textbf{\textcolor{black}{5}}, \textbf{\textcolor{black}{5}}, \textbf{\textcolor{black}{6}}, \textbf{\textcolor{black}{6}}, \textbf{\textcolor{black}{8}}, \textbf{\textcolor{black}{8}}, \textbf{\textcolor{black}{9}}, \textbf{\textcolor{black}{10}}, \textbf{\textcolor{black}{10}}, \textbf{\textcolor{black}{13}}, \textbf{\textcolor{black}{14}}, \textbf{\textcolor{black}{14}}, \textbf{\textcolor{black}{21}}
\end{center}

\begin{enumerate}[A]
\item 11
\item 9.5
\item 8
\item 9
\item 10
\end{enumerate}

\begin{multicols}{2}
\null \vfill
\vfill \null
\columnbreak
\includegraphics[width = 0.35\textwidth]{../slide_images/iClicker_logo.png}

\end{multicols}

\end{frame}

\begin{frame}{iClicker quiz: Question \textcolor{blue}{3}}

What is the \textbf{3rd Quartile}, also known as the \textbf{75th
percentile}, rounded to the nearest integer, of the following sequence
of numbers?

\textcolor{black}{0}, \textcolor{black}{1}, \textcolor{black}{2},
\textcolor{black}{3}, \textcolor{black}{4}, \textcolor{black}{5},
\textcolor{black}{6}, \textcolor{black}{7}, \textcolor{black}{8},
\textcolor{black}{9}, \textcolor{black}{10}, \textcolor{black}{11},
\textcolor{black}{12}

\begin{enumerate}[A]
\item 6
\item 7
\item 9
\item 11
\item 12
\end{enumerate}

\begin{multicols}{2}
\null \vfill
\vfill \null
\columnbreak
\includegraphics[width = 0.35\textwidth]{../slide_images/iClicker_logo.png}

\end{multicols}

\end{frame}

\begin{frame}{Announcements}

\begin{block}{Salamander groups}

\begin{itemize}
\tightlist
\item
  \textbf{\textcolor{red}{Some students were not included in salamander description groups.}}
\item
  \textbf{\textcolor{red}{Please verify your group membership and check the Moodle gradebook.}}
\end{itemize}

\end{block}

\begin{block}{Bear peer-feedback forms}

\begin{itemize}
\tightlist
\item
  Extended until tomorrow night.
\item
  Future peer-feedback form deadlines will be enforced.
\end{itemize}

\end{block}

\end{frame}

\begin{frame}{Announcements}

This is a short week.

Chapter 4 has a lot of important information.

We'll continue chapter 4 materials into next week.

The pre-class exercises for next week will include reading questions
from chapters 4 and 5.

Short lecture today: I want to have time for the in-class group
activity.

\end{frame}

\begin{frame}[fragile]{Questions from the salamander exercise:}

\begin{itemize}
\tightlist
\item
  What is SVL?
\item
  Why is there a \texttt{\$} in \texttt{mander}?
\item
  How is \emph{central tendency} related to the \emph{spread}?
\item
  How do we define \emph{quartiles}?
\item
  Loading data files into R
\item
  In-class R instruction
\end{itemize}

\end{frame}

\begin{frame}{Sample Statistics}

What are two ways to summarize a collection of numbers?

\begin{itemize}
\tightlist
\item
  Central tendency
\item
  Dispersion
\end{itemize}

Why do we call these \emph{statistics}?

\end{frame}

\begin{frame}{Distributions}

What are key features of the \textbf{Normal Distribution}?

What is the \textbf{Uniform Distribution}?

\end{frame}

\begin{frame}{Group Activity: Random numbers}

Self-select groups of 3 or 4.

Follow the instructions on Moodle.

Submit a single report for the group.

\end{frame}

\end{document}
